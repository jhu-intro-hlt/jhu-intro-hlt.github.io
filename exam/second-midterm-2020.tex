\documentclass[12pt]{article}
\usepackage{palatino}
\usepackage{fullpage}
\usepackage{graphicx}
\usepackage{enumitem}
\usepackage{multirow}
\usepackage{latexsym}
\usepackage{amsmath,amssymb}
\usepackage[utf8]{inputenc}

\setlength{\parindent}{0in}
\setlength{\parskip}{.1in}

\newcounter{QuestionCounter}
%\newcommand{\Question}[1]{\refstepcounter{QuestionCounter} {\bf Question \arabic{QuestionCounter}}. [#1 points] }

\newcommand{\Question}[2]{\refstepcounter{QuestionCounter} \textbf{\Large {Q\arabic{QuestionCounter}}. #1\hfill\textit{#2 points}}}

\begin{document}

{\ }
\vskip 1in

\centerline{\Huge\bf Second Midterm Exam}
\vskip .5in
\centerline{\LARGE\bf 601.467/667 Introduction to Human Language Technology}
\vskip .2in
\centerline{\Large Fall 2020}
\vskip .1in
\centerline{\Large Johns Hopkins University}
\vskip .1in
\centerline{\Large Co-ordinator: Philipp Koehn}
\vskip .3in
\centerline{\Large 10 November 2020}
\vskip 1in
\centerline{\Large Complete all questions.}
\vskip .2in
\centerline{\Large Use additional paper if needed.}
\vskip .2in
\centerline{\Large Time: 75 minutes.}
\vskip 1in
\centerline{\Large Name of student: {\rule{3in}{0.4pt}}}

\newpage
\Question{\Large Auditory System and Speech Basics}{30}
\vskip .25in
\begin{enumerate}
\item What are the approximate estimates of bit rates of speech signal and of linguistic message in speech and how would you estimate it ?
\vskip 2.3in

\item Why do human speakers generate different speech sounds?
\vskip 2.3in

\item What is an equal loudness curve and what does it tell us about how the brain perceives sounds?
\vskip 2.3in

\item Explain how the brain converts sound waves into neural signals? 
\vskip 3in

\item You were asked to develop a technology for a deaf person where you need to bypass damaged parts of their ear. You are focusing mostly on speech-based communication (less so on music and other sounds). Knowing what you know about both speech signals and the auditory system, and assuming no biological or computational/engineering constraints, what can you do to deliver a speech signal to help this person? Which parts of the signal are critical? 
\vskip 2in
\end{enumerate}

\newpage
\Question{Spectrograms}{10}
\vskip .25in

Recall that the spectrogram is a visual representation of $X[k,\omega]$ of a speech signal $x[n]$ in time-window $n\in[k-\Delta,k+\Delta]$ centered at $k$, with $k$ represented along the $x$-axis of the spectrogram, $\omega$ along the $y$-axis, and the magnitude of the Fourier transform $|X[k,\omega]|$ plotted as the pixel intensity, darker for higher values.
\begin{enumerate}
    \item What window sizes (in milliseconds) are associated with a {\em narrow-band} versus a {\em wide-band} spectrogram?  What is commonly used for automatic speech recognition?
    \vskip 1.25in
    \item What is the typical range of frequencies (in Hz or kHz) on the $y$-axis in which speech signals have spectral energies?  How does it depend on the sampling rate of $x[n]$?
    \vskip 1.25in
    \item What is a {\em formant}, and how it is visually manifested in a spectrogram?  What kinds of phonemes typically generate formants?
    \vskip 1.25in
    \item How does one identify fricatives sounds (e.g. an /sh/ or /zh/) in a spectrogram?
    \vskip 1.25in
\end{enumerate}

\newpage

\Question{Mel-Frequency Cepstral Coefficients}{10}
\vskip .25in

Recall that a typical signal representation used for automatic speech recognition is a sequence of Mel-Frequency Cepstral Coefficients or MFCCs.  In what order are the following signal processing operations carried out to compute MFCCs?
\def\msquare{\mathord{\scalebox{1.5}[1]{\scalerel*{\Box}{\strut}}}}
\begin{description}
    \item[\Huge $\Box$] Discrete cosine transform
    \item[\Huge $\Box$] Discrete Fourier transform
    \item[\Huge $\Box$] Logarithm computation
    \item[\Huge $\Box$] Magnitude computation
    \item[\Huge $\Box$] Mel-weighting (triangular filters of increasing width)
    \item[\Huge $\Box$] Preemphasis
    \item[\Huge $\Box$] Sampling the continuous-time signal
    \item[\Huge $\Box$] Windowing
\end{description}
State the psycho-acoustic motivation for
\begin{itemize}
    \item ignoring the phase of the Fourier transform,
    \item taking the logarithm, and
    \item the increasing width of the Mel-filters.
\end{itemize}

\newpage

\Question{Hidden Markov Models}{10}
\vskip .25in

A major reason for the popularity of hidden Markov models (HMMs) for acoustic modeling is their compositional nature: HMMs for phonemes can be strung together to create HMMs for words, which can be strung together to create HMMs for sentences, etc.

Using ${\bf x}$ to denote a sequence of observations $x_1,\ldots,x_T$ from an HMM, ${\bf s}$ the sequence $s_1,\ldots,s_T$ of unobserved states, and $s_0$ the (known) initial state, one may write
\begin{equation*}
    P({\bf x},{\bf s}\,|\,s_0) \,\,\,=\,\,\, \prod_{n=1}^T P(x_n|s_n)\,P(s_n|s_{n-1}).
\end{equation*}
State, both in words and in a mathematical precise expression, the computational problem associated with HMMs that is solved by
\begin{enumerate}
    \item the forward-backward algorithm,
    \vskip 1.7in
    \item the Baum-Welch algorithm, and
    \vskip 1.7in
    \item the Viterbi algorithm.
\end{enumerate}
\newpage
\begin{itemize}
    \item For each algorithm above, specify its computational (big-O) complexity in terms of the input length $T$, and the number of states $S$ in the HMM.  
    \vskip 2in
    \item How are $E$ and $S$ related for a general HMM versus a left-to-right HMM?  Refine your answer above in terms of the number of permitted transitions (edges) $E$ in the HMM instead of the number of states $S$.
    \vskip 2in
    \item What is the typical value of $S$ in a large vocabulary speech recognition system?
    \begin{enumerate}
        \item Ten to a hundred
        \item Hundred to a thousand
        \item Thousand to ten thousand
        \item Ten thousand to a million
        \item More than a million
    \end{enumerate}
    What is the typical value of $E$?
    \vskip 2in
\end{itemize}

\newpage





\newpage
\Question{Speaker Recognition}{20}
\vskip .25in
What is the role of linear discriminant analysis or probabilistic linear discriminant analysis applied to speech representation for speaker recognition?

\newpage
\Question{Neural Speech Recognition}{20}
\vskip .25in
Please discuss the difference between HMM-based, connectionist temporal classification (CTC), and attention-based speech recognition systems in terms of 1) conditional independence assumptions, 2) use of language and lexicon/pronunciation models, 3) implementation, 4) speech recognition performance. You can also include the other perspectives if you want (e.g., training cost and number of parameters, as discussed in our class).

\begin{itemize}
\item conditional independence assumptions (including Markov assumptions)
\vspace{2.5cm}
\item use of language and lexicon/pronunciation models
\vspace{2.5cm}
\item implementation 
\vspace{2.5cm}
\item speech recognition performance
\vspace{2.5cm}
\item others? (optional)
\end{itemize}


\end{document}

